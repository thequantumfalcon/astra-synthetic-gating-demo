\appendix
\section{Appendix A: operator viewpoint}
\label{app:operator}

This appendix expands on the operator viewpoint used in the main text.

Let $x \in \mathbb{R}^N$ be a discrete time series. Define the hard-gating operator $g_\tau$ elementwise by
\begin{equation}
 (g_\tau(x))_k = x_k\,\mathbb{I}(|x_k| \le \tau).
\end{equation}

\subsection{Nonlinearity}
$g_\tau$ is nonlinear. In particular, for general $x,y$ and scalar $\alpha$,
\begin{equation}
 g_\tau(x+y) \ne g_\tau(x) + g_\tau(y),\qquad g_\tau(\alpha x) \ne \alpha g_\tau(x).
\end{equation}

This is why it is useful to make the operator explicit before discussing any downstream statistic.

\subsection{Effect on extrema}
Define $M(x)=\max_k |x_k|$. For any $x$,
\begin{equation}
 M(g_\tau(x)) \le \tau.
\end{equation}

Thus, if a downstream score depends directly on $M(x)$ (or is strongly correlated with it), then gating can bound that score.

\subsection{A simple bound for peak-based proxies}
If a noise scale estimate $\sigma>0$ is fixed, then the peak proxy $S(x)=M(x)/\sigma$ obeys
\begin{equation}
 S(g_\tau(x)) \le \tau/\sigma.
\end{equation}

The toy experiment in this submission simply instantiates these relationships with explicit arrays and logs.

\section{Methods: synthetic injection and gating}
\label{sec:methods}

This section defines the minimal experiment implemented by \texttt{astra\_proof.py}.

\subsection{Signal construction}
We construct a time grid $t \in [0, T]$ sampled at $f_s$ Hz. A synthetic transient $h(t)$ is injected at time $t_0$ with a narrow envelope. The precise functional form is specified by the code and is part of the reproducibility contract.

The observed series is
\begin{equation}
\label{eq:observed}
 x(t) = h(t) + n(t),
\end{equation}
where $n(t)$ is additive, zero-mean Gaussian noise with standard deviation $\sigma$.

\subsection{Gating operator}
We apply a simple amplitude threshold gate. In the abstract, a gating operator takes an input series $x(t)$ and produces an altered series $g(x)(t)$ by applying a rule such as
\begin{equation}
\label{eq:gating}
 g(x)(t) = \begin{cases}
 0, & |x(t)| > \tau \\
 x(t), & \text{otherwise,}
 \end{cases}
\end{equation}
where $\tau$ is a threshold. For convenience we report the \emph{gated fraction}
\medskip

\noindent Formally, the gate is just an elementwise nonlinear operator; it is used here as a transparent toy model of amplitude-based transient mitigation.
\begin{equation}
\label{eq:gatedfrac}
 f_g = \frac{1}{N}\sum_{k=1}^N \mathbb{I}(|x_k| > \tau),
\end{equation}
where $x_k$ denotes the discrete samples.

\subsection{Summary statistics}
We compute simple before/after summary statistics that are easy to verify:
\begin{itemize}
\item Peak SNR proxy before gating: $\max_t |x(t)| / \sigma$.
\item Peak SNR proxy after gating: $\max_t |g(x)(t)| / \sigma$.
\item Gated fraction $f_g$.
\end{itemize}

These are not meant to replicate matched filtering or Bayesian inference; they are chosen for transparency.

\subsection{Determinism and artifacts}
For a fixed seed, the script writes:
\begin{itemize}
\item \texttt{verification\_log.txt}: a human-readable report of the parameters and measured statistics.
\item \texttt{astra\_injection.npz}: arrays $t$, $x$, and $g(x)$.
\end{itemize}

Optionally, with \texttt{--mc N}, the script runs $N$ trials (seed sweep) and writes:
\begin{itemize}
\item \texttt{mc\_summary.csv}: per-trial statistics.
\item \texttt{mc\_table.tex}: a compact LaTeX table with quantiles.
\end{itemize}

\subsection{Proposed test on real/open strain (falsifiable protocol)}
The synthetic experiment above establishes a mechanism: gating can suppress a burst-shaped transient prior to scoring. To make the hypothesis falsifiable on real data \emph{when and if the relevant segment is public}, the bundle includes \texttt{astra\_real\_verify.py}, which implements an explicit protocol with a real/open-data attempt (via \texttt{gwpy}) and a documented synthetic fallback.

The proposed test can be stated as a checklist:
\begin{enumerate}
\item Select a target GPS time $t_\mathrm{gps}$ and detector(s) (e.g., H1/L1) and fetch a window of strain around $t_\mathrm{gps}\pm 32\,$s if open data are available.
\item Define an injected or hypothesized burst template (frequency, damping time, and amplitude scale used for the protocol comparison).
\item Compute a proxy recovery score on the \emph{ungated} stream and on a \emph{gated} stream produced by a clearly specified rule.
\item Report the ratio of ungated-to-gated scores, and archive the full code, parameters, and logs.
\end{enumerate}

\noindent \textbf{Pass/fail criterion (for the gating-suppression prediction):} the prediction is supported if, under the specified rule, the ungated stream yields a meaningfully larger proxy recovery score than the gated stream (e.g., a multi-$\times$ ratio), and this behavior is stable under small perturbations of the gating parameters. The prediction is falsified if no such suppression is observed under reasonable gating rules.

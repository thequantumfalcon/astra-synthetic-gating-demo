\section{Discussion}
\label{sec:discussion}

\subsection{What this does and does not show}
This submission shows that a deterministic gating operator can strongly alter peak-based summary metrics in a constructed injection-and-noise experiment. It does \emph{not} show:
\begin{itemize}
\item that any particular astrophysical model is correct,
\item that any particular detector observed a signal,
\item that gating is universally beneficial or harmful,
\item or that peak-SNR proxies are sufficient for inference.
\end{itemize}

The motivating context discussed in Section~\ref{sec:background} is included only to motivate an illustrative burst morphology for the synthetic capsule and should not be read as an observational or model-selection claim.

The strongest claim supported by the included artifacts is a \emph{mechanism claim}: given an injected transient and a specified gating rule, the gate can reduce a peak-based proxy score by an order of magnitude. This motivates an empirical check on real data conditioning, but it is not itself a detection.

\subsection{Why include a paper at all?}
The manuscript exists to make the artifact self-describing:
\begin{itemize}
\item It states assumptions explicitly.
\item It defines the operator and metrics.
\item It describes the outputs and how to regenerate them.
\end{itemize}

\subsection{Potential extensions (not implemented here)}
A reader could extend this capsule in several directions without changing the core idea:
\begin{enumerate}
\item Replace peak-based metrics with a matched-filter statistic against a known template.
\item Replace Gaussian noise with colored noise and include whitening.
\item Explore threshold schedules (time-varying $\tau(t)$).
\item Replace hard gating with soft clipping (e.g., $\tanh$) and compare.
\end{enumerate}

We intentionally do not include these extensions to keep the core script small and reviewable.

\subsection{Proposed real-data test (protocol)}
To make the prediction falsifiable, the bundle includes \texttt{astra\_real\_verify.py}. When open data are available, it attempts to fetch public strain for a user-specified GPS time and detector using standard tooling (\texttt{gwpy}). If open data are not accessible, it runs an equivalent synthetic protocol and writes a log file that documents parameters and proxy outcomes.

This design provides reviewers with an immediate, concrete path to test the hypothesis on public data when/if the relevant segment becomes available, without implying unauthorized access to restricted archives.
